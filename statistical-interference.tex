% Options for packages loaded elsewhere
\PassOptionsToPackage{unicode}{hyperref}
\PassOptionsToPackage{hyphens}{url}
%
\documentclass[
]{article}
\usepackage{amsmath,amssymb}
\usepackage{lmodern}
\usepackage{ifxetex,ifluatex}
\ifnum 0\ifxetex 1\fi\ifluatex 1\fi=0 % if pdftex
  \usepackage[T1]{fontenc}
  \usepackage[utf8]{inputenc}
  \usepackage{textcomp} % provide euro and other symbols
\else % if luatex or xetex
  \usepackage{unicode-math}
  \defaultfontfeatures{Scale=MatchLowercase}
  \defaultfontfeatures[\rmfamily]{Ligatures=TeX,Scale=1}
\fi
% Use upquote if available, for straight quotes in verbatim environments
\IfFileExists{upquote.sty}{\usepackage{upquote}}{}
\IfFileExists{microtype.sty}{% use microtype if available
  \usepackage[]{microtype}
  \UseMicrotypeSet[protrusion]{basicmath} % disable protrusion for tt fonts
}{}
\makeatletter
\@ifundefined{KOMAClassName}{% if non-KOMA class
  \IfFileExists{parskip.sty}{%
    \usepackage{parskip}
  }{% else
    \setlength{\parindent}{0pt}
    \setlength{\parskip}{6pt plus 2pt minus 1pt}}
}{% if KOMA class
  \KOMAoptions{parskip=half}}
\makeatother
\usepackage{xcolor}
\IfFileExists{xurl.sty}{\usepackage{xurl}}{} % add URL line breaks if available
\IfFileExists{bookmark.sty}{\usepackage{bookmark}}{\usepackage{hyperref}}
\hypersetup{
  hidelinks,
  pdfcreator={LaTeX via pandoc}}
\urlstyle{same} % disable monospaced font for URLs
\usepackage[margin=1in]{geometry}
\usepackage{graphicx}
\makeatletter
\def\maxwidth{\ifdim\Gin@nat@width>\linewidth\linewidth\else\Gin@nat@width\fi}
\def\maxheight{\ifdim\Gin@nat@height>\textheight\textheight\else\Gin@nat@height\fi}
\makeatother
% Scale images if necessary, so that they will not overflow the page
% margins by default, and it is still possible to overwrite the defaults
% using explicit options in \includegraphics[width, height, ...]{}
\setkeys{Gin}{width=\maxwidth,height=\maxheight,keepaspectratio}
% Set default figure placement to htbp
\makeatletter
\def\fps@figure{htbp}
\makeatother
\setlength{\emergencystretch}{3em} % prevent overfull lines
\providecommand{\tightlist}{%
  \setlength{\itemsep}{0pt}\setlength{\parskip}{0pt}}
\setcounter{secnumdepth}{-\maxdimen} % remove section numbering
\ifluatex
  \usepackage{selnolig}  % disable illegal ligatures
\fi

\author{}
\date{\vspace{-2.5em}}

\begin{document}

\hypertarget{exponential-distribution-compare-to-central-limit-theorem}{%
\section{Exponential Distribution compare to Central limit
Theorem}\label{exponential-distribution-compare-to-central-limit-theorem}}

\hypertarget{the-project-consists-of-two-parts}{%
\section{The project consists of two
parts:}\label{the-project-consists-of-two-parts}}

\hypertarget{a-simulation-exercise.}{%
\section{1. A simulation exercise.}\label{a-simulation-exercise.}}

\hypertarget{basic-inferential-data-analysis.}{%
\section{2. Basic inferential data
analysis.}\label{basic-inferential-data-analysis.}}

\hypertarget{a-simulation-exercise}{%
\section{1: A simulation exercise}\label{a-simulation-exercise}}

\hypertarget{overview}{%
\section{Overview}\label{overview}}

\hypertarget{in-this-project-the-exponential-distribution-is-investigated-in-r-and-compare-it-with-central-limit-theorem.-the-mean-of-exponential-distribution-is-1lambda-and-the-standard-deviation-is-also-a-function-of-1lambda.-the-exponential-distribution-is-simulated-in-r-with-rexpnlambda-where-lambda0.2-for-all-of-the-simulations-sample-size-n-40-and-the-number-of-simulation-1000.}{%
\section{In this project the exponential distribution is investigated in
R and compare it with Central Limit Theorem. The mean of exponential
distribution is 1/lambda and the standard deviation is also a function
of 1/lambda. The exponential distribution is simulated in R with
rexp(n,lambda), where lambda=0.2 for all of the simulations, sample size
n = 40, and the number of simulation
=1000.}\label{in-this-project-the-exponential-distribution-is-investigated-in-r-and-compare-it-with-central-limit-theorem.-the-mean-of-exponential-distribution-is-1lambda-and-the-standard-deviation-is-also-a-function-of-1lambda.-the-exponential-distribution-is-simulated-in-r-with-rexpnlambda-where-lambda0.2-for-all-of-the-simulations-sample-size-n-40-and-the-number-of-simulation-1000.}}

\hypertarget{simulations}{%
\section{Simulations}\label{simulations}}

\hypertarget{a-series-of-1000-simulations-is-run-to-create-a-data-set-for-comparison-purpose.-each-simulation-contain-40-observations-and-the-expoential-distribution-function-will-be-set-to-rexp40-0.2-where-0.2-is-lambda-value.}{%
\section{A series of 1000 simulations is run to create a data set for
comparison purpose. Each simulation contain 40 observations and the
expoential distribution function will be set to rexp(40, 0.2) where 0.2
is lambda
value.}\label{a-series-of-1000-simulations-is-run-to-create-a-data-set-for-comparison-purpose.-each-simulation-contain-40-observations-and-the-expoential-distribution-function-will-be-set-to-rexp40-0.2-where-0.2-is-lambda-value.}}

\hypertarget{given-data-n-40-simnum-1000-lambda-0.2}{%
\section{Given data: n = 40; simNum = 1000; lambda =
0.2}\label{given-data-n-40-simnum-1000-lambda-0.2}}

\hypertarget{for-reproducibility-set-seed-10000}{%
\section{For reproducibility, set seed =
10000}\label{for-reproducibility-set-seed-10000}}

\hypertarget{exponential-sampling-parameters}{%
\section{Exponential sampling
parameters}\label{exponential-sampling-parameters}}

\hypertarget{set-seed-for-reproducability}{%
\section{set seed for
reproducability}\label{set-seed-for-reproducability}}

set.seed(31) \# set lambda to 0.2 lambda \textless- 0.2 \# 40 samples n
\textless- 40 \# 1000 simulations simulations \textless- 1000 \#
simulate simulated\_exponentials \textless- replicate(simulations,
rexp(n, lambda)) \# calculate mean of exponentials means\_exponentials
\textless- apply(simulated\_exponentials, 2, mean)

\#Question 1

\#Show where the distribution is centered at and compare it to the
theoretical center of the distribution. analytical\_mean \textless-
mean(means\_exponentials) analytical\_mean \# analytical mean
theory\_mean \textless- 1/lambda theory\_mean \# visualization
meanSample = mean(simul\(Mean) meanTheory = 1/lambda hist(simul\)Mean,
breaks = 30, prob = TRUE,col = ``lightblue'', main=``Exponential
Distribution of Sample Means'', xlab=``Means of 40 Simulated Samples'',
ylab = ``Counts'') abline(v = meanTheory, col= ``red'', lwd = 3)
abline(v = meanSample, col = ``blue'',lwd = 2) legend(`topright',
c(``Sample Mean'', ``Theoretical Mean''), bty = ``n'',\\
lty = c(1,1), col = c(col = ``blue'', col = ``red'')) \# The analytics
mean is 5.006 the theoretical mean 5. The center of distribution of
averages of 40 exponentials is very close to the theoretical center of
the distribution.

\hypertarget{question-2}{%
\section{Question 2}\label{question-2}}

\hypertarget{show-how-variable-it-is-and-compare-it-to-the-theoretical-variance-of-the-distribution..}{%
\section{Show how variable it is and compare it to the theoretical
variance of the
distribution..}\label{show-how-variable-it-is-and-compare-it-to-the-theoretical-variance-of-the-distribution..}}

\hypertarget{standard-deviation-of-distribution}{%
\section{standard deviation of
distribution}\label{standard-deviation-of-distribution}}

standard\_deviation\_dist \textless- sd(means\_exponentials)
standard\_deviation\_dist

\hypertarget{standard-deviation-from-analytical-expression}{%
\section{standard deviation from analytical
expression}\label{standard-deviation-from-analytical-expression}}

standard\_deviation\_theory \textless- (1/lambda)/sqrt(n)
standard\_deviation\_theory

\hypertarget{variance-of-distribution}{%
\section{variance of distribution}\label{variance-of-distribution}}

variance\_dist \textless- standard\_deviation\_dist\^{}2 variance\_dist
\# variance from analytical expression variance\_theory \textless-
((1/lambda)*(1/sqrt(n)))\^{}2 variance\_theory

\hypertarget{standard-deviation-of-the-distribution-is-0.7931608-with-the-theoretical-sd-calculated-as-0.7905694.-the-theoretical-variance-is-calculated-as-1-1n2-0.625.-the-actual-variance-of-the-distribution-is-0.6291041}{%
\section{Standard Deviation of the distribution is 0.7931608 with the
theoretical SD calculated as 0.7905694. The Theoretical variance is
calculated as ((1 / ??) * (1/???n))2 = 0.625. The actual variance of the
distribution is
0.6291041}\label{standard-deviation-of-the-distribution-is-0.7931608-with-the-theoretical-sd-calculated-as-0.7905694.-the-theoretical-variance-is-calculated-as-1-1n2-0.625.-the-actual-variance-of-the-distribution-is-0.6291041}}

\hypertarget{question-3}{%
\section{Question 3}\label{question-3}}

\hypertarget{show-that-the-distribution-is-approximately-normal.}{%
\section{Show that the distribution is approximately
normal.}\label{show-that-the-distribution-is-approximately-normal.}}

xfit \textless- seq(min(means\_exponentials), max(means\_exponentials),
length=100) yfit \textless- dnorm(xfit, mean=1/lambda,
sd=(1/lambda/sqrt(n)))
hist(means\_exponentials,breaks=n,prob=T,col=``skyblue'',xlab =
``means'',main=``Density of means'',ylab=``density'') lines(xfit, yfit,
pch=22, col=``red'', lty=5) \# compare the distribution of averages of
40 exponentials to a normal distribution qqnorm(means\_exponentials)
qqline(means\_exponentials, col = 2) \# Due to Due to the central limit
theorem (CLT), the distribution of averages of 40 exponentials is very
close to a normal distribution. \# 2. Basic inferential data analysis.
\# Load the ToothGrowth data and perform some basic exploratory data
analyses \# load the data ToothGrowth data(ToothGrowth) \# preview the
structure of the data str(ToothGrowth) \# preview first 5 rows of the
data head(ToothGrowth, 5) \# Provide a basic summary of the data. \#
data summary summary(ToothGrowth) \# compare means of the different
delivery methods tapply(ToothGrowth\(len,ToothGrowth\)supp, mean) \#
plot data graphically library(ggplot2) ggplot(ToothGrowth,
aes(factor(dose), len, fill = factor(dose))) + geom\_boxplot() + \#
facet\_grid(.\textasciitilde supp)+ facet\_grid(.\textasciitilde supp,
labeller = as\_labeller( c(``OJ'' = ``Orange juice'', ``VC'' =
``Ascorbic Acid''))) + labs(title = ``Tooth growth of 60 guinea pigs by
dosage and\nby delivery method of vitamin C'', x = ``Dose in
milligrams/day'', y = ``Tooth Lengh'') + scale\_fill\_discrete(name =
``Dosage of\nvitamin C\nin mg/day'') + theme\_classic() \# Use
confidence intervals and/or hypothesis tests to compare tooth growth by
supp and dose. \# comparison by delivery method for the same dosage t05
\textless- t.test(len \textasciitilde{} supp, data =
rbind(ToothGrowth{[}(ToothGrowth\(dose == 0.5) &  (ToothGrowth\)supp ==
``OJ''),{]},
ToothGrowth{[}(ToothGrowth\(dose == 0.5) &  (ToothGrowth\)supp ==
``VC''),{]}), var.equal = FALSE)

t1 \textless- t.test(len \textasciitilde{} supp, data =
rbind(ToothGrowth{[}(ToothGrowth\(dose == 1) &  (ToothGrowth\)supp ==
``OJ''),{]},
ToothGrowth{[}(ToothGrowth\(dose == 1) &  (ToothGrowth\)supp ==
``VC''),{]}), var.equal = FALSE)

t2 \textless- t.test(len \textasciitilde{} supp, data =
rbind(ToothGrowth{[}(ToothGrowth\(dose == 2) &  (ToothGrowth\)supp ==
``OJ''),{]},
ToothGrowth{[}(ToothGrowth\(dose == 2) &  (ToothGrowth\)supp ==
``VC''),{]}), var.equal = FALSE)

\hypertarget{summary-of-the-conducted-t.tests-which-compare-the-delivery-methods-by-dosage}{%
\section{summary of the conducted t.tests, which compare the delivery
methods by
dosage,}\label{summary-of-the-conducted-t.tests-which-compare-the-delivery-methods-by-dosage}}

\hypertarget{take-p-values-and-ci}{%
\section{take p-values and CI}\label{take-p-values-and-ci}}

summaryBYsupp \textless- data.frame( ``p-value'' =
c(t05\(p.value, t1\)p.value,
t2\(p.value),  "Conf.Low" = c(t05\)conf.int{[}1{]},t1\(conf.int[1], t2\)conf.int{[}1{]}),
``Conf.High'' = c(t05\(conf.int[2],t1\)conf.int{[}2{]},
t2\$conf.int{[}2{]}), row.names = c(``Dosage .05'',``Dosage 1'',``Dosage
2''))

\hypertarget{show-data-table}{%
\section{show data table}\label{show-data-table}}

summaryBYsupp \# Conclusion \# For dosage of .5 milligrams/day and 1
milligrams/day does matter the delivery method. the delivery method for
2 milligrams/day. For dosage of 2 milligrams/day the delivery method
doesn't matter.

\end{document}
